\documentclass{article}
\usepackage{amsmath}

\author{Tawit Uthaicharoenpong}
\title{UKF notes}
\date{\today}

\begin{document}
\maketitle

This document is to note down what I have done during working on this assignment.

\section{UKF initialization}
\subsection{Variance and Standard deviation}
First thing we need to do is initialize the UKF class.
Process noise in this project consisted of:

\begin{itemize}
    \item longitudinal accelation (aka. linear accelation), represent as variance \(\sigma^2_a\) 
    with unit \( \frac{m^2}{s^4}\) or as standard deviation \(\sigma_a\) with unit \(\frac{m}{s^2}\)

    \item yaw accelation (aka. angular acceleration), represent as variance \(\sigma^2_{\ddot{\varphi}}\)
    with unit \(\frac{rad^2}{s^4}\) or as standard deviation \(\sigma_{\ddot{\varphi}}\) with unit \(\frac{rad}{s^2}\)
\end{itemize}

In the lesson, suggested starting value for linear accelation varince is \(\sigma^2_a = 9 \frac{m^2}{s^4}\) which means we expected linear accelation within range \(\pm 2\sigma_a\) or from \(-6\frac{m}{s^2}\) to \(6\frac{m}{s^2} \rightarrow 
\sigma_a = 3\).\\

For angular acceleration I would start with \(\sigma^2_{\ddot{\varphi}}\) = 1.5\(\frac{rad^2}{s^4}\) or from 
\(-1.22\frac{rad}{s^2}\) to \(1.22\frac{rad}{s^2} \rightarrow \sigma_{\ddot{\varphi}}=1.22\).\\

% End subsection: Variance and Standard deviation

\subsection{State dimension and lambda}
\[
State vector, x =
\begin{bmatrix}
p_x \\ p_y \\ v \\ \varphi \\ \ddot{\varphi}
\end{bmatrix}
\]

Therefore, number of state, $n_x = 5$.\\

For augmented state, we taken process noise into account:\\
\[
Augmented state vector, x_{aug} =
\begin{bmatrix}
p_x \\ p_y \\ v \\ \varphi \\ \ddot{\varphi} \\ \nu_a \\ \nu_{\ddot{\varphi}}
\end{bmatrix}
\]
Therefore, number of augmented state dimention, $n_{aug} = 7$.\\

Lambda, $\lambda$, is a design parameter with a rule-of-thumbs $\lambda = 3 - n_x$

% End subsection: State dimension

\subsection{Accents}
\begin{equation} 
    a^2 + b^2 = c^2 test
\end{equation}

\subsection{Dollar signs}

    
\end{document}